% Options for packages loaded elsewhere
\PassOptionsToPackage{unicode}{hyperref}
\PassOptionsToPackage{hyphens}{url}
%
\documentclass[
]{article}
\usepackage{amsmath,amssymb}
\usepackage{lmodern}
\usepackage{iftex}
\ifPDFTeX
  \usepackage[T1]{fontenc}
  \usepackage[utf8]{inputenc}
  \usepackage{textcomp} % provide euro and other symbols
\else % if luatex or xetex
  \usepackage{unicode-math}
  \defaultfontfeatures{Scale=MatchLowercase}
  \defaultfontfeatures[\rmfamily]{Ligatures=TeX,Scale=1}
\fi
% Use upquote if available, for straight quotes in verbatim environments
\IfFileExists{upquote.sty}{\usepackage{upquote}}{}
\IfFileExists{microtype.sty}{% use microtype if available
  \usepackage[]{microtype}
  \UseMicrotypeSet[protrusion]{basicmath} % disable protrusion for tt fonts
}{}
\makeatletter
\@ifundefined{KOMAClassName}{% if non-KOMA class
  \IfFileExists{parskip.sty}{%
    \usepackage{parskip}
  }{% else
    \setlength{\parindent}{0pt}
    \setlength{\parskip}{6pt plus 2pt minus 1pt}}
}{% if KOMA class
  \KOMAoptions{parskip=half}}
\makeatother
\usepackage{xcolor}
\IfFileExists{xurl.sty}{\usepackage{xurl}}{} % add URL line breaks if available
\IfFileExists{bookmark.sty}{\usepackage{bookmark}}{\usepackage{hyperref}}
\hypersetup{
  pdftitle={Lesson 2},
  pdfauthor={Nicholas Clark},
  hidelinks,
  pdfcreator={LaTeX via pandoc}}
\urlstyle{same} % disable monospaced font for URLs
\usepackage[margin=1in]{geometry}
\usepackage{color}
\usepackage{fancyvrb}
\newcommand{\VerbBar}{|}
\newcommand{\VERB}{\Verb[commandchars=\\\{\}]}
\DefineVerbatimEnvironment{Highlighting}{Verbatim}{commandchars=\\\{\}}
% Add ',fontsize=\small' for more characters per line
\usepackage{framed}
\definecolor{shadecolor}{RGB}{248,248,248}
\newenvironment{Shaded}{\begin{snugshade}}{\end{snugshade}}
\newcommand{\AlertTok}[1]{\textcolor[rgb]{0.94,0.16,0.16}{#1}}
\newcommand{\AnnotationTok}[1]{\textcolor[rgb]{0.56,0.35,0.01}{\textbf{\textit{#1}}}}
\newcommand{\AttributeTok}[1]{\textcolor[rgb]{0.77,0.63,0.00}{#1}}
\newcommand{\BaseNTok}[1]{\textcolor[rgb]{0.00,0.00,0.81}{#1}}
\newcommand{\BuiltInTok}[1]{#1}
\newcommand{\CharTok}[1]{\textcolor[rgb]{0.31,0.60,0.02}{#1}}
\newcommand{\CommentTok}[1]{\textcolor[rgb]{0.56,0.35,0.01}{\textit{#1}}}
\newcommand{\CommentVarTok}[1]{\textcolor[rgb]{0.56,0.35,0.01}{\textbf{\textit{#1}}}}
\newcommand{\ConstantTok}[1]{\textcolor[rgb]{0.00,0.00,0.00}{#1}}
\newcommand{\ControlFlowTok}[1]{\textcolor[rgb]{0.13,0.29,0.53}{\textbf{#1}}}
\newcommand{\DataTypeTok}[1]{\textcolor[rgb]{0.13,0.29,0.53}{#1}}
\newcommand{\DecValTok}[1]{\textcolor[rgb]{0.00,0.00,0.81}{#1}}
\newcommand{\DocumentationTok}[1]{\textcolor[rgb]{0.56,0.35,0.01}{\textbf{\textit{#1}}}}
\newcommand{\ErrorTok}[1]{\textcolor[rgb]{0.64,0.00,0.00}{\textbf{#1}}}
\newcommand{\ExtensionTok}[1]{#1}
\newcommand{\FloatTok}[1]{\textcolor[rgb]{0.00,0.00,0.81}{#1}}
\newcommand{\FunctionTok}[1]{\textcolor[rgb]{0.00,0.00,0.00}{#1}}
\newcommand{\ImportTok}[1]{#1}
\newcommand{\InformationTok}[1]{\textcolor[rgb]{0.56,0.35,0.01}{\textbf{\textit{#1}}}}
\newcommand{\KeywordTok}[1]{\textcolor[rgb]{0.13,0.29,0.53}{\textbf{#1}}}
\newcommand{\NormalTok}[1]{#1}
\newcommand{\OperatorTok}[1]{\textcolor[rgb]{0.81,0.36,0.00}{\textbf{#1}}}
\newcommand{\OtherTok}[1]{\textcolor[rgb]{0.56,0.35,0.01}{#1}}
\newcommand{\PreprocessorTok}[1]{\textcolor[rgb]{0.56,0.35,0.01}{\textit{#1}}}
\newcommand{\RegionMarkerTok}[1]{#1}
\newcommand{\SpecialCharTok}[1]{\textcolor[rgb]{0.00,0.00,0.00}{#1}}
\newcommand{\SpecialStringTok}[1]{\textcolor[rgb]{0.31,0.60,0.02}{#1}}
\newcommand{\StringTok}[1]{\textcolor[rgb]{0.31,0.60,0.02}{#1}}
\newcommand{\VariableTok}[1]{\textcolor[rgb]{0.00,0.00,0.00}{#1}}
\newcommand{\VerbatimStringTok}[1]{\textcolor[rgb]{0.31,0.60,0.02}{#1}}
\newcommand{\WarningTok}[1]{\textcolor[rgb]{0.56,0.35,0.01}{\textbf{\textit{#1}}}}
\usepackage{graphicx}
\makeatletter
\def\maxwidth{\ifdim\Gin@nat@width>\linewidth\linewidth\else\Gin@nat@width\fi}
\def\maxheight{\ifdim\Gin@nat@height>\textheight\textheight\else\Gin@nat@height\fi}
\makeatother
% Scale images if necessary, so that they will not overflow the page
% margins by default, and it is still possible to overwrite the defaults
% using explicit options in \includegraphics[width, height, ...]{}
\setkeys{Gin}{width=\maxwidth,height=\maxheight,keepaspectratio}
% Set default figure placement to htbp
\makeatletter
\def\fps@figure{htbp}
\makeatother
\setlength{\emergencystretch}{3em} % prevent overfull lines
\providecommand{\tightlist}{%
  \setlength{\itemsep}{0pt}\setlength{\parskip}{0pt}}
\setcounter{secnumdepth}{-\maxdimen} % remove section numbering
\ifLuaTeX
  \usepackage{selnolig}  % disable illegal ligatures
\fi

\title{Lesson 2}
\author{Nicholas Clark}
\date{}

\begin{document}
\maketitle

\hypertarget{admin}{%
\subsection{Admin}\label{admin}}

\vspace{.3in}

The begining of our text focuses on experiments vs.~observational
studies. Why is this important?

\vspace{.3in}

At West Point, as well as at most universities, prior to conducting an
experiment, your \textbf{study protocol} must be reviewed by an
Institutional Review Board or IRB. The point of the IRB is to protect
the rights of the subjects of a study as well as to ensure that
inferences made from the study are statistically valid.

A \textbf{double blind} study is:

\vspace{.4in}

Why is this important?

\vspace{.5in}

Our book talks about a study on store ratings and wants to determine
whether a rating is influenced by exposure to a scent. Are there ethical
issues with this study?

\vspace{.3in}

The first model they consider is

\begin{align*}
i &= \mbox{ Student}\\
y_i & = \mbox{rating of student }i\\
y_i & = \mu + \epsilon_i
\end{align*}

What is the sources of variation diagram associated with this model?

\vspace{2.in}

What does \(\epsilon_i\) represent in this model?

\vspace{1.3in}

Are there any assumptions we are making on \(\epsilon_i\)?

\vspace{.5in}

The book says that the fitted model is:

\begin{align*}
y_i &= 4.48 + \epsilon_i\\
\epsilon_i & \sim F(0,1.27)
\end{align*}

Note here I use the generic \(F\) to stand for some distribution, I'm
not making any distributional assumptions on \(\epsilon_i\). How did the
book find \(\hat{\mu}=4.48\) and the standard error of the residiuals as
1.27?

\vspace{1.5in}

What assumption are we making when we use this model? What would our
causal diagram look like?

\vspace{.5in}

What is the treatment variable? Let's sketch out the sources of
variation diagram

\vspace{1.4in}

Our proposed diagram is:

\vspace{1.in}

We can visualize:

\begin{Shaded}
\begin{Highlighting}[]
\FunctionTok{library}\NormalTok{(tidyverse)}

\NormalTok{dat}\OtherTok{=}\FunctionTok{read.table}\NormalTok{(}\StringTok{"http://www.isi{-}stats.com/isi2/data/OdorRatings.txt"}\NormalTok{,}\AttributeTok{header=}\NormalTok{T)}

\NormalTok{dat }\SpecialCharTok{\%\textgreater{}\%} \FunctionTok{ggplot}\NormalTok{(}\FunctionTok{aes}\NormalTok{(}\AttributeTok{x=}\NormalTok{condition, }\AttributeTok{y=}\NormalTok{rating,}\AttributeTok{fill=}\NormalTok{condition)) }\SpecialCharTok{+} 
  \FunctionTok{geom\_violin}\NormalTok{(}\AttributeTok{trim =} \ConstantTok{FALSE}\NormalTok{)}\SpecialCharTok{+}
  \FunctionTok{geom\_dotplot}\NormalTok{(}\AttributeTok{binaxis=}\StringTok{\textquotesingle{}y\textquotesingle{}}\NormalTok{, }\AttributeTok{stackdir=}\StringTok{\textquotesingle{}center\textquotesingle{}}\NormalTok{)}\SpecialCharTok{+}
  \FunctionTok{coord\_flip}\NormalTok{()}
\end{Highlighting}
\end{Shaded}

\includegraphics{Lsn_3_files/figure-latex/unnamed-chunk-1-1.pdf}

A statistical model that could be used to address the scientific
question is:

\vspace{1.5in}

How could we fit this model? Well, getting the estimates for \(\mu_1\)
and \(\mu_2\) shouldn't be hard.

\begin{Shaded}
\begin{Highlighting}[]
\NormalTok{dat }\SpecialCharTok{\%\textgreater{}\%} \FunctionTok{group\_by}\NormalTok{(condition)}\SpecialCharTok{\%\textgreater{}\%}\FunctionTok{summarize}\NormalTok{(}\AttributeTok{samp.mus=}\FunctionTok{mean}\NormalTok{(rating),}\AttributeTok{sds=}\FunctionTok{sd}\NormalTok{(rating))}
\end{Highlighting}
\end{Shaded}

\begin{verbatim}
## # A tibble: 2 x 3
##   condition samp.mus   sds
##   <chr>        <dbl> <dbl>
## 1 noscent       3.83 1.24 
## 2 scent         5.12 0.947
\end{verbatim}

\begin{Shaded}
\begin{Highlighting}[]
\NormalTok{scent.model}\OtherTok{=}\FunctionTok{lm}\NormalTok{(rating}\SpecialCharTok{\textasciitilde{}}\DecValTok{0}\SpecialCharTok{+}\NormalTok{condition,}\AttributeTok{data=}\NormalTok{dat)}
\FunctionTok{summary}\NormalTok{(scent.model)}
\end{Highlighting}
\end{Shaded}

\begin{verbatim}
## 
## Call:
## lm(formula = rating ~ 0 + condition, data = dat)
## 
## Residuals:
##     Min      1Q  Median      3Q     Max 
## -1.8333 -0.8333 -0.1250  0.8750  2.1667 
## 
## Coefficients:
##                  Estimate Std. Error t value Pr(>|t|)    
## conditionnoscent   3.8333     0.2251   17.03   <2e-16 ***
## conditionscent     5.1250     0.2251   22.76   <2e-16 ***
## ---
## Signif. codes:  0 '***' 0.001 '**' 0.01 '*' 0.05 '.' 0.1 ' ' 1
## 
## Residual standard error: 1.103 on 46 degrees of freedom
## Multiple R-squared:  0.9461, Adjusted R-squared:  0.9438 
## F-statistic:   404 on 2 and 46 DF,  p-value: < 2.2e-16
\end{verbatim}

Note that the standard error from this output does not match the
standard error given on the top of page 39. Why do you think that is?
How could we match the standard error given on page 39?

\vspace{1.5in}

Looking at the output, (ignoring p values for now), what appears to be
happening? How certain are we? How could we be sure?

\vspace{.5in}

What could be a confounding variable for this study?

\vspace{.5in}

The most important part of thinking of confounding is given in figure
1.1.5.

\vspace{.5in}

This is in our text, but it bears repeating: The goal of random
assignment is to reduce the chances of there being any confounding
variables in the study. By creating groups that are expected to be
similar with respect to all variables (other than the treatment variable
of interest) that may impact the response, random assignment attempts to
eliminate confounding. A key consequence of not having variables
confounded with the treatment variable in a randomized experiment is the
potential to draw cause-and-effect conclusions between the treatment
variable and the response variable.

\url{https://www.vox.com/science-and-health/2018/6/20/17464906/mediterranean-diet-science-health-predimed}

\hypertarget{think---if-our-investigators-wanted-to-know-if-there-was-a-difference-between-scent-and-noscent-what-would-we-be-testing-in-terms-of-our-parameters}{%
\subsection{\texorpdfstring{Think - If our investigators wanted to know
if there was a difference between \texttt{scent} and \texttt{noscent}
what would we be testing in terms of our
parameters?}{Think - If our investigators wanted to know if there was a difference between scent and noscent what would we be testing in terms of our parameters?}}\label{think---if-our-investigators-wanted-to-know-if-there-was-a-difference-between-scent-and-noscent-what-would-we-be-testing-in-terms-of-our-parameters}}

\end{document}
